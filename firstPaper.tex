
\documentclass[aps,floatfix,prd,showpacs]{revtex4}
%\documentclass[aps,floatfix,prd,showpacs,twocolumn]{revtex4}
\usepackage{graphicx}% Include figure files
\usepackage{dcolumn}% Align table columns on decimal point
\usepackage{bm}% bold math

\voffset 1.0cm

\begin{document}


\title{Comparison of Computer Simulation Methods for Predicting Chemical Reactions}
\author{Graham Gibson}
\affiliation{
Department of Physics and Astronomy,
University of Pittsburgh,
Pittsburgh, PA 15260,
USA.}

\date{\today}

\begin{abstract}
In this paper we compare two primary methods of predicting basic organic chemistry reaction predictions. We analyze two types of models, NLP based Neural Network methods and agent based modeling. We compare and contrast the complexity, accuracy, and generalizability of both models.  
\end{abstract}
\pacs{PACS numbers go here. These are classification codes for your  research. See {\tt http://publish.aps.org/PACS/} for more info.}
\maketitle

\section{Introduction}

Using latex is pretty easy if you have a sample document you can follow.

\section{Results}
Including figures, tables, and equations is easy. Latex also permits easy reference to document elements (figures, tables, sections) with the \begin{verbatim}\ref\end{verbatim} command\ref{fig1}. Citations are made with the \begin{verbatim}\cite\end{verbatim} command\cite{lamport}. 

\begin{figure}[ht]
%\includegraphics[width=7cm,angle=-90]{linear_q_eq_0.ps}
\caption{You will need to include the package graphicx to be able to make figures like this.}
\label{fig1}
\end{figure}

A simple table.

\begin{table}[ht]
\caption{$X(3872)$ Discovery Modes.}
\label{XmodesTab}
\begin{tabular}{cclccl}
\hline
mass & width & production/decay mode & events & significance & experiment\\
\hline
\hline
$3872.0 \pm 0.6 \pm 0.5$  & $< 2.3$ 90\% C.L.  & $B^\pm \to K^\pm X \to K^\pm \pi^+ \pi^- J/\psi$   &  $25.6 \pm 6.8$ & $10 \sigma$     & Belle\\
$3871.3 \pm 0.7 \pm 0.4$  & resolution & $p\bar p \to  X \to \pi^+ \pi^- J/\psi$   &  $730 \pm 90$ & $11.6 \sigma$  & CDFII\\
$M(J/\psi) + 774.9 \pm 3.1 \pm 3.0$ & resolution & $p\bar p \to X \to \pi^+\pi^-J/\psi$ & $522 \pm 100$ & $5.2 \sigma$  & D{\O} \\
$3873.4 \pm 1.4$  &  --  & $B^- \to K^- X \to K^- \pi^+ \pi^- J/\psi$   &  $25.4 \pm 8.7$ &$3.5 \sigma$ & BaBar\\
\hline
\hline
\end{tabular}
\end{table}

And a sample equation (Eq. \ref{XCD}).

\begin{equation}
\Gamma(X \to \alpha\beta D) = \int {d^3Q\over (2\pi)^3}  \Gamma(C\to \alpha\beta) {|\tilde T(Q)|^2 \over
(M(X) - E_{CD}(Q))^2 + \Gamma_C^2/4}
\label{XCD}
\end{equation}





\section{Conclusions}

Man, latex is great!

\acknowledgments
The author is grateful to Donald Knuth for inventing tex, and making publication quality typesetting a reality for scientists around the world.


\begin{thebibliography}{99}

\bibitem{lamport}
 {\sl LaTeX : A Documentation Preparation System User's Guide and Reference Manual}, Leslie Lamport [1994] (ISBN: 0-201-52983-1) pages: xvi+272.

\bibitem{latt}
I.M. Smart {\it et al.}, J. Plumb Phys. {\bf 50}, 393 (1983).

\end{thebibliography}

\end{document}



